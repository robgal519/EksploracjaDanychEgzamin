\documentclass[a4paper,12pt]{article}

\usepackage{indentfirst}
\usepackage{cite}
\usepackage[utf8]{inputenc}
\usepackage[T1]{polski}
\usepackage{helvet}
\usepackage{graphicx}
\usepackage{svg}
\usepackage{color}
\usepackage{geometry}
\usepackage{float}
\usepackage{multirow}
\usepackage[hidelinks]{hyperref}
\usepackage{caption}
% dodaj bibliografię do spisu treści
\usepackage[nottoc]{tocbibind}
% unikaj pojedynczych linii na początku/ końcu stron
\usepackage[defaultlines=4,all]{nowidow}
\usepackage{subcaption}
\usepackage{minted}
\usepackage{amsmath}
\usepackage{amsfonts}

% set default figure placement to !htbp
\makeatletter
\def\fps@figure{!htbp}
\makeatother


\setminted{ frame=lines,
framesep=2mm,
baselinestretch=1.2,
fontsize=\footnotesize,
linenos
}

\newcommand{\setSubtitle}[1]{
    \newcommand{\subtitle}{#1}
    }


    
    \usepackage[most]{tcolorbox}

    %textmarker style from colorbox doc
    \tcbset{textmarker/.style={%
            enhanced,
            parbox=false,boxrule=0mm,boxsep=0mm,arc=0mm,
            outer arc=0mm,left=6mm,right=3mm,top=7pt,bottom=7pt,
            toptitle=1mm,bottomtitle=1mm,oversize}}
    
    
    % define new colorboxes
    \newtcolorbox{hintBox}{textmarker,
        borderline west={6pt}{0pt}{yellow},
        colback=yellow!10!white}
    \newtcolorbox{importantBox}{textmarker,
        borderline west={6pt}{0pt}{red},
        colback=red!10!white}
    \newtcolorbox{noteBox}{textmarker,
        borderline west={6pt}{0pt}{green},
        colback=green!10!white}
    
    % define commands for easy access
    \newcommand{\note}[1]{\begin{noteBox} \textbf{Note:} #1 \end{noteBox}}
    \newcommand{\warning}[1]{\begin{hintBox} \textbf{Warning:} #1 \end{hintBox}}
    \newcommand{\important}[1]{\begin{importantBox} \textbf{Important:} #1 \end{importantBox}}

        \usepackage{xcolor}
    