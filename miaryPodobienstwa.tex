\section{Miary (nie) podobieństwa}\label{sec:metryki}
Odelgłość elementów $a,b \in X$ jest wartością funkcji $d(a,b)$ nazywanej funkcją odległości takiej że:
$$
d:X x X \rightarrow [0,\infty]
$$
 spełniającą warunki:
 \begin{itemize}
     \item $\forall_{a,b \in X} [d(a,a) \le d(a,b)] $
     \item $\forall_{a,b \in X} [d(a,b) = d(b,a)] $
     \item $\forall_{a,b,z \in X} [d(a,z) \le d(a,b)+d(b,z)] $
 \end{itemize}

Funkcje odległości:

dla ilości wymiaró równej $1$:

metryka Manhatan ( k = 1) czyli moduł różnicy, lub semimetryka:
$$
d(a,b) = (a-b)^2
$$

dla ilości wymiarów większej niż $1$:
\begin{itemize}
    \item \textbf{Norma $L_2$} -- Euklidesowa (z wykładu) $$ \sqrt{\sum_{i=1}^k(x_i - y_i)^2} $$
    \item \textbf{Norma $L_1$} -- Manhattan/taksówkowa (z wykładu) $$ \sum_{i=1}^k|x_i - y_i|$$
    \item \textbf{Norma $L_{\infty}$} -- max z wymiarów $$ max_{i=1}^k |x_i-y_i| $$ 
    \item semimetryka -- ( z wykładu) $$ d(a,b) = \sum_{i=1}^k (a_i-b_i)^2$$
    \item Odległość Mińkowskiego $$ (\sum_{i=1}^k(|x_i-y_i)^q)^{\frac{1}{q}} $$
    \item Odległość Canberra -- Ważona wersja odległości Manhattan, stosowana dla danych o zliczeniach (np. o rozkładzie zbliżonym do Poissona), uporządkowanych rankingach itp. $$ \sum_{i=1}^n \cfrac{x_i-y_i}{|x_i|+|y_i|} $$
    \item Odległość Hamminga -- Liczba różnych współrzędnych. Często wykorzystywana dla ciągów cyfr lub liter, ale też dla binarnych wektorów. 
\end{itemize} 

Możemy je stosować jako \textbf{miary (nie) podobieństwa}. 